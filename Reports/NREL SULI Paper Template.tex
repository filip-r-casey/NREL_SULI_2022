\documentclass[12pt, oneside]{article}   	% use "amsart" instead of "article" for AMSLaTeX format
\usepackage{geometry}                		% See geometry.pdf to learn the layout options. There are lots.
\geometry{letterpaper, left=1in, right=1in, top=1in, bottom=1in}                   		% ... or a4paper or a5paper or ... 
\usepackage{changepage}
\usepackage{graphicx}				% Use pdf, png, jpg, or eps§ with pdflatex; use eps in DVI mode
								% TeX will automatically convert eps --> pdf in pdflatex		
\usepackage{amssymb}
\usepackage{sectsty}
\usepackage{indentfirst}
\usepackage{lipsum}
\usepackage{amsmath}
\usepackage{etoolbox}
\usepackage[format=plain, font=it]{caption}

\apptocmd{\thebibliography}{\raggedright}{}{}
\allsectionsfont{\normalsize\mdseries}
\sectionfont{\normalsize\mdseries\MakeUppercase}
\renewcommand{\thesection}{\Roman{section}.} 
\renewcommand{\thesubsection}{\Alph{subsection}.}
\renewcommand{\thesubsubsection}{\arabic{subsubsection}.}

\usepackage[superscript,biblabel]{cite}


\begin{document}

\begin{titlepage}
	\centering

	\textbf{Communities to Clean: Wind Resource Data Aggregation Tool}\par
	\vspace{14pt}
	Filip Casey\par
	\vspace{42pt}
	Office of Science, Science Undergraduate Laboratory Internship Program\par
	\vspace{14pt}
	University of Colorado, Boulder, CO\par
	\vspace{42pt}
	National Renewable Energy Laboratory\par
	\vspace{14pt}
	Golden, Colorado\par
	\vspace{56pt}
	\today\par
	\vspace{70pt}
	Prepared in partial fulfillment of the requirement of the Department of
	Energy, Office of Science's Science Undergraduate Laboratory Internship
	Program under the direction of MENTOR at the National Renewable Energy
	Laboratory.\par
	\vspace{56pt}
	Participant: \line(1,0) {100}\par
	\hspace{63pt} Author\par
	\vspace{28pt}
	Mentor: \line(1,0) {100}\par
	\hspace{40pt} Mentor\par
	\vfill
	\thispagestyle{plain}
\end{titlepage}

\setcounter{page}{2}

\section*{Abstract}
\begin{adjustwidth}{.5in}{.5in}
	As focus on renewable energy increases, the ability to quickly plan and
	develop projects, small and large, is becoming more necessary. The
	Communities to Clean (C2C) project's goal is to present community leaders
	with easily accessible data to promote the adoption of renewable energy
	projects.

	Previously, when starting a renewable energy project, its leaders have had
	to do intensive, site-specific research to understand how successful a plant
	can be. This research is often lengthy and expensive, leading to significant
	knowledge gaps and limiting accessibility for smaller communities. The C2C
	project aims to enable these community leaders by collecting, analyzing, and
	presenting essential data, through a free, open-source data platform.

	Currently, the project is focused on wind energy, sourcing data from NASA
	POWER, NREL's Wind Toolkit, Open Weather, National Weather Service, and the
	Alaska Energy Authority. This toolkit allows for users to easily access
	visualizations and raw data through a GUI and enables developers to access
	years of data, from multiple sources, through a single API call. This data
	platform was also built with modularity in mind, with future goals to
	implement solar energy and hybrid optimization. By simplifying access and
	analysis, this project seeks to empower those promoting renewable energy
	development.
\end{adjustwidth}

\section{introduction}

The primary focus of the Communities to Clean project is presenting community
leaders with easily accessible data that can enable them to site renewable
energy projects without an extensive monetary commitment. For this to be
successful, users need to have access to a variety of data sources that can
present a comprehensive view of a certain site. As a result, this report aims to
offer an in-depth comparison of the possible data sources, an overview of the
implementation of this project, and a perspective on the future possibilities of
this platform.

The data sources that will be compared in this report are NASA POWER, the
Department of Energy's Wind Toolkit, Open Weather, the National Weather
Service's API, and meteorologic tower data from the Alaska Energy Authority.
These sources were each selected because they offered an advantage does not
present in the others. Although they each have downfalls, the hope of this
project is that through the presentation of multiple public data sources, with
clearly analyzed assumptions, renewable energy siting can be done significantly
quicker and cheaper.

This platform consists of a user-friendly interface focused on satisfying the
needs of community leaders, as well as an API to provide efficient developer
access. The platform was created in nodejs using the express framework, and the
user interface was created using the ejs framework. The Alaska Energy Authority
API included in this project was created in the same manner, using a postgresql
database to store the data scraped the Alaska Energy Authority.

This platform will become a potent tool for planning hybrid renewable energy
solutions in the future. Currently, wind power is implemented, but the next step
is taking advantage of the numerous publicly available solar resources. On top
of this, a consumer-friendly comparison of these data sources will allow for
greater usability of the tool and allow for the Communities to Clean project to
become a serious solution for renewable energy siting.

\section{data}
\subsection{NASA POWER}
NASA's Prediction Of Worldwide Energy Resources (POWER) aims to provide public
access to predictive meteorologic data \cite{NASA-POWER-HOME}. POWER leverages
NASA's extensive data collection and processing abilities to extrapolate
meteorologic information for any point on earth. This makes it an incredibly
versatile data source, that can help intimate information for more remote
locations that other data sources may miss.

It offers a spatial resolution of .5° x .625°, which is incredibly low.
Depending on the location, meteorologic tower data can often be found within
this range, which has the benefit of being empirically collected data. However,
the advantage of the NASA data is that its vertical interpolation, to account
for hub height, is done within a consistent model, where meteorologic tower data
would require its own interpolation.

Its temporal resolution is an hourly average, which is standard for these data
sources and has data from “1981 to within several months of real
time.”\cite{NASA-POWER-METHODOLOGY}  This means that it has the greatest data
availability of any of these sources, largely thanks to the strength of NASA's
predictions. The other major advantage of this source is that it comes with a
built-in corrected wind speed model. \cite{NASA-POWER-SPEED} Users can input the
terrain type of the site that they are considering, and NASA will use this
information to more accurately predict how the wind speed will change with hub
height. This is significantly more accurate than the other vertical
interpolation models that will be used, so it is a huge advantage of this data
source.

Overall NASA POWER is the most comprehensive of these sources, offering a large
variety of data, with the major caveat that little to none of it is empirically
collected. The accuracy is entirely dependent on NASA's models and the
undisclosed availability of information that they have.

\subsection{Wind Toolkit}

The Department of Energy's Wind Toolkit operates in a similar manner, instead
using data from wind turbines to source more empirical power
information.\cite{WIND-PAPER} The extrapolations for this source are slightly
more limited, likely to account for the greater accuracy it can provide. It only
has data for the continental United States and is incredibly limited in Alaska
(citation needed).

However, these limitations mean that it can provide the highest spatial and
temporal resolution. It has a spatial resolution of 2km x 2km (or about 0.018° x
0.018°), and a temporal resolution of down to a 5-minute average, which is
something that none of the other data sources can offer. This is somewhat
limited by the data only being available from 2007-2013, but still provides
incredibly specific information if it's required.

Its vertical interpolation is based on a different model than NASA's Corrected
Wind Speed and doesn't account for terrain type. This means that it could be
slightly less accurate, but for higher hub heights, it should be reasonably
irrelevant. It provides information for the discrete hub heights of 10, 40, 60,
80, 100, 120, 140, 160 and 200m, but if a more specific value was required, it
could be done with a simple calculation.

For data limited to the US, Wind Toolkit is one of the most comprehensive
sources, with some small technical limitations. Due to the large nature of the
data, instead of responding to an API call with a JSON or csv, a download link
is provided with an email being sent to the requester when the data is finished
being analyzed. While this isn't the largest obstacle in aggregating this data,
it makes it significantly harder to present in a front-end web application and
will make it significantly more inaccessible to community leaders. On top of
this, each API call should provide all required data in one call, which means
that rate limiting isn't a major issue. The caveat here is that the request size
is limited instead by the defined request weight.

\begin{figure}[h]
	\[\text{Request Weight}=\text{site-count}*\text{attribute-count}*\text{year-count}*\text{data-intervals-per-year} \]
	\caption[Wind Toolkit Request Weight Equation]{Request Weight, Wind Toolkit allows for a max request weight of 175,000,000. Site count is based on the amount of sites containing data for a certain point. Attribute count is the number of data points requested. Year count is how many years of data were requested. Data intervals per year is how many points of data are being requested every year.}
\end{figure}

If used properly, Wind Toolkit is a strong dataset for siting renewable energy
projects in the US. It has the same limitations as other model-based data
sources, but the advantage of being significantly more accurate.

\subsection{Open Weather}
Open Weather is an Open-Source alternative to Weather Underground, both
platforms focus on the collection of data from Personal Weather Stations (PWS)
and meteorological extrapolation methods to provide accurate localized weather
data.

There are notable benefits to this type of data. Primarily, the temporal
resolution goes down to a minute, so it is the only service that can provide
near real-time data. It also has 40+ years of historical data thanks to its
extrapolation models. On top of this the spatial resolution is exact.

However, these benefits come with some significant downfalls. First, this falls
into the same category as NASA POWER, since it is heavily relying on
extrapolated data. The primary difference is that POWER has the advantage of
being designed specifically for these metrics, with corrected wind speed
parameters, that Open Weather does not have. This means that vertical
interpolation would have to be done after Open Weather's model has done its
work, further distorting the raw data. Open Weather's main benefit, and reason
for inclusion, is the sheer number of data sources. Open Weather has 80,000
personal weather stations that it draws information from, and this can make it
an incredibly versatile source.\cite{OPEN-HOME} Although there is no geographic
information provided about these stations (which side of a house it's on) it can
still provide more reliable information where other sources have less
availability.

Another important aspect to note is that the API requires a call for each time
step. On top of this, a user key has only 1,000 free calls per day. If Open
Weather will be included in the platform, it will be essential to discuss how
this implementation can work best for all stakeholders.

Open Weather is not a perfect source, but it is an essential one, providing
information where it may not be available from others, at the cost of some
accuracy.

\subsection{National Weather Service}
The National Weather Service (NWS) API provides access to meteorologic data from
all NWS stations. When a point is provided, raw data from any of the stations
within the area is returned.

The NWS does not use any extrapolation models, so the spatial resolution is
based entirely on the proximity of a station. The use of empirical station data
also means that there is some variance in data availability. The temporal
resolution seems to vary from station to station, with all of them at least
being less than an hour. One of the major downfalls of this data source is that
the only data publicly available is very close to the current date. This makes
it incredibly hard to compare to the other data sources, but possibly makes it
more valuable for getting a better understanding of current data than sources
like Alaska Energy Authority and Wind Toolkit whose data is near 10 years old at
this point.

The fact that there is only about a month of data available at any time is a
severe downfall and the lack of any vertical interpolation model means that it
must be done manually, but having up-to-date information is still incredibly
important. This data could also possibly be archived to create an accurate and
modern long-term data source that isn't available elsewhere. Overall, despite
its shortcomings this is a data source worth including.

\subsection{Alaska Energy Authority}
Perhaps the most limited in scope, this data source still offers incredibly
potent information by focusing on an incredibly valuable area in renewable
energy that has a serious lack of data availability in other sources. Wind
Toolkit is a very helpful tool for the mainland US but has lots of missing
information in Alaska. The Alaska Energy Authority has no established API, but
one was developed to enable easier data access for this project.

This information is again a station-based data source which means that the
spatial resolution is exact, but reliant on where stations are located. One of
the most significant downfalls, however, is its temporal resolution. At best,
(when analyzing raw data) data is recorded hourly. Most of the data presented is
more summary based and even less precise than that. On top of this, all of the
data available ranges from the 1970s to the early 2000s, there is no modern data
available from this source.

This makes it slightly less useful for actual data use but provides a potent
comparison for the rest of the data sources. The Alaska Energy Authority acts as
a keystone for the data comparison in this report as it provides the ability to
compare the model-based data sources to an empirical data source in an area that
historically has less data availability.

\section{methods}
\subsection{Software Development}
\subsubsection{Functionality}
This software consists of a front-end for user access and a back-end for
developers and to serve content.

The back-end was created in the ExpressJS framework for nodejs. This framework
allows for a very simple, unopinionated approach to designing
APIs.\cite{EXPRESS-HOME} At each endpoint, the backend creates a request
specific to each data source based on the uniformly inputted user data. Then
once the response is received from each of the sources, they are parsed in order
to present all of the data in the same format, making it easier to understand
for community leaders and developers alike.

The front-end was developed using the VueJS framework, which simplifies the
creation of single page web interfaces. This reduces the load on the server, as
it requires less data to be processed. The front-end uses a simple form, that
allows for users to input their desired parameters, and sends it to the backend.
This calls the API, which will then return the data to be served to the webpage.
It can display it in a tabular format, and download a csv if desired.

Lastly, within this project, there is also a newly created endpoint for Alaska
Energy Authority Data. As previously dicussed, the Alaska Energy Authority is an
important source, but its data was only presented in formatted text files. In
order to make this more usable, I wrote python scripts that could scrape this
data, format it and store it in a postgresql database. Then in the server,
instead of calling an API to return Alaska Energy Authority data, user inputs
are formatted into SQL commands to return the data directly.
\subsubsection{Maintenance}
In order for this project to have long-term viability, extensive support and
testing infrastructure needed to be built in from day one. Any of the APIs that
this project accesses are subject to change at any time, package updates could
deprecate features, and long-term use could reveal bugs that couldn't be
discovered during the development process.

As a result, tests have been implemented for each endpoint using the
jest/supertest framework. This allows for faux api calls to be made, so that it
can be ensured that the response from the app is as expected. Tests for error
handling, formatting, and correct responses were written for each endpoint.
These tests were then implemented into the Github continuous integration
platform, Github Actions, so that every time the software is changed, it can be
automatically tested that it is in working order.

On the user-side, extensive API documentation was created using the OpenAPI
specification to provide a comprehensive guide to using the tool. This
documentation highlights parameters and their meanings, expected response
schemas, and an oppurtunity to try API requests. It can also be hosted in the
same place as the front and back-end, so the users have all the information and
tools that they need in one place.
\subsection{Data Analysis}
Due to the limited amount of raw wind data, an extensive data-focused comparison of the each of the sources was not possible for the this project. However, there are still some metrics that were used, and the possibility for a more comprehensive analysis in the future.

\section{Results}
\subsection{Software Development}
\subsection{Data Analysis}
\section{Discussion}
\section{Conclusion}
\section{Acknowledgments}

This work was supported in part by the U.S. Department of Energy, Office of
Science, Office of Workforce Development for Teachers and Scientists (WDTS)
under the Science Undergraduate Laboratory Internships Program (SULI).

\begin{thebibliography}{99}

	\bibitem{NASA-POWER-HOME}
	"NASA POWER | Prediction Of Worldwide Energy Resources." NASA,
	https://power.larc.nasa.gov/. Accessed 21 June 2022.
	\bibitem{NASA-POWER-METHODOLOGY}
	NASA POWER | Docs | Methodology - NASA POWER | Docs.
	https://power.larc.nasa.gov/docs/methodology/. Accessed 21 June 2022.
	\bibitem{NASA-POWER-SPEED}
	Wind Speed - NASA POWER | Docs.
	https://power.larc.nasa.gov/docs/methodology/meteorology/wind/. Accessed 21
	June 2022.
	\bibitem{WIND-PAPER}
	Draxl, Caroline, et al. “The Wind Integration National Dataset (WIND)
	Toolkit.” Applied Energy, vol. 151, Aug. 2015, pp. 355-66. ScienceDirect,
	https://doi.org/10.1016/j.apenergy.2015.03.121.
	\bibitem{OPEN-HOME}
	“Current Weather and Forecast - OpenWeatherMap.” OpenWeather,
	https://openweathermap.org/. Accessed 5 July 2022.
	\bibitem{EXPRESS-HOME}
	Express - Node.Js Web Application Framework. https://expressjs.com/.
	Accessed 1 Aug. 2022.
\end{thebibliography}


\end{document}